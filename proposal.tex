\documentclass{article}

\begin{document}
\section{Introduction}
According to the United Nations, urbanization is accelerating. In 2008, for the first time in humankind's history, more people lived in cities than did in rural areas. The number of people living in cities will only increase in the 21st century. To meet this challenge, urban planners are turning to "smart" solutions to the problems associated with modern cities. One such problem is that of traffic congestion. Congestion is strongly correlated to population density \cite{Manville2005}, and is thus only expected to increase with growing urbanization. A "smart" solution is to use increased connectivity of vehicles to improve individuals' paths through the traffic network. Particularly we focus on inter-vehicular communication, and how that can be used to improve traffic routing.

\section{Problem description}
Reducing congestion, or improving throughput requires a multi-pronged approach by urban planners. In the 2010 International Transport Forum, the OECD highlighted managing congestion as one of the key challenges for the future in transportation, saying: 
\begin{quote}
\emph{Better managing congestion, including new mechanisms that lead towards more efficient use of network capacity, ensuring strong linkages between land use and transport planning, and applying innovative traveller information and traffic management systems supported by mobile communications and other technologies.} \cite{OECD2010}
\end{quote}

In other words, congestion management can be improved by more efficient use of the existing infrastructure, better zoning to ensure short travel times and innovations in traffic through the use of, specifically, information technologies. Other organizations, both governmental and industrial, focus on similar solutions. For instance, IBM states that:

\begin{quote}
\emph{At IBM, we see five broad categories of solution to the problem of traffic:	
\begin{enumerate}
\item better route guidance, to use roads more efficiently      
\item intelligent transportation systems, including better traffic prediction, to allow people to alter their routes or traveling times and allow system operators to manage the road network better      
\item greater ease in switching between cars and various forms of public transportation      
\item faster removal of road blockages due to damages or collisions from location-­based information      
\item more dynamic workplaces that allow telecommuting flexibility      
\end{enumerate}
} \cite{IBM2011}
\end{quote}

In this project we aim to address the first and second points, by addressing how inter-vehicular communication can be used to improve routing. Inter-vehicular communication allows vehicles to provide each other with information about the state of the road and traffic. This can be unsollicited information, or vehicles can ask specifically for pertinent information. Obtained information can be used to better model the current situation, which is useful in path planning and traffic prediction. However, there are a number of open issues that must be resolved before inter-vehicular communication can truly be used in this capacity. We divide these issues into roughly two groups: deciding what information should be communicated, and deciding what to do with received information. In reality, however, these groups show considerable overlab. For instance, when asking for specific information, it is necessary to know what information is needed, which in turn depends on what it is being used for.

\subsection{What should communicated?}

\subsection{What  should be done with received information?}

\section{State of the Art}
\section{Objectives and Workplan}
We each write our own workplan here

\bibliographystyle{plain}
\bibliography{proposal}
\end{document}

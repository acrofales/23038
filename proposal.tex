\documentclass{article}

%\usepackage{pgf}
%\usepackage{tikz}
%\usetikzlibrary{arrows,positioning}

\begin{document}
\section{Introduction}
According to the United Nations, urbanization is accelerating. In 2008, for the first time in humankind's history, more people lived in cities than did in rural areas. The number of people living in cities will only increase in the 21st century. To meet this challenge, urban planners are turning to "smart" solutions to the problems associated with modern cities. One such problem is that of traffic congestion. Congestion is strongly correlated to population density \cite{Manville2005}, and is thus only expected to increase with growing urbanization. A "smart" solution is to use increased connectivity of vehicles to improve individuals' paths through the traffic network. Particularly we focus on inter-vehicular communication, and how that can be used to improve traffic routing.

\section{Problem description}
Reducing congestion, or improving throughput requires a multi-pronged approach by urban planners. In the 2010 International Transport Forum, the OECD highlighted managing congestion as one of the key challenges for the future in transportation, saying: 
\begin{quote}
\emph{Better managing congestion, including new mechanisms that lead towards more efficient use of network capacity, ensuring strong linkages between land use and transport planning, and applying innovative traveller information and traffic management systems supported by mobile communications and other technologies.} \cite{OECD2010}
\end{quote}

In other words, congestion management can be improved by more efficient use of the existing infrastructure, better zoning to ensure short travel times and innovations in traffic through the use of, specifically, information technologies. Other organizations, both governmental and industrial, focus on similar solutions. For instance, IBM states that:

\begin{quote}
\emph{At IBM, we see five broad categories of solution to the problem of traffic:	
\begin{enumerate}
\item better route guidance, to use roads more efficiently      
\item intelligent transportation systems, including better traffic prediction, to allow people to alter their routes or traveling times and allow system operators to manage the road network better      
\item greater ease in switching between cars and various forms of public transportation      
\item faster removal of road blockages due to damages or collisions from location-­based information      
\item more dynamic workplaces that allow telecommuting flexibility      
\end{enumerate}
} \cite{IBM2011}
\end{quote}

In this project we aim to address the first and second points, by addressing how inter-vehicular communication can be used to improve routing. Inter-vehicular communication allows vehicles to provide each other with information about the state of the road and traffic. This can be unsollicited information, or vehicles can ask specifically for pertinent information. Obtained information can be used to better model the current situation, which is useful in path planning and traffic prediction. However, there are a number of open issues that must be resolved before inter-vehicular communication can truly be used in this capacity. We divide these issues into roughly two groups: deciding what information should be communicated, and deciding what to do with received information. In reality, however, these groups show considerable overlab. For instance, when asking for specific information, it is necessary to know what information is needed, which in turn depends on what it is being used for.

\subsection{What should be communicated?}
Not all information needs to be communicated. In fact, the Braess paradox \cite{Braess1969} shows that individual drivers might sometimes be better off having incomplete, or even false, information. The Braess paradox describes a simple road network, in which the construction of a fast road between two points decreases overall throughput in the system. This same could be applied to knowledge of the state of that road. Throughput could be increased by spreading false information that a road is blocked, or less controversially, withholding information that a road is not blocked. 

This is merely one example of withholding information leading to better overall performance. Another example is that of people learning that the main road between two points is congested and taking the alternative. However, if the alternative cannot handle the increased amounts of traffic and the resulting traffic jam is worse than the one on the main road. Drivers would have been better off not knowing the main road was congested in the first place.

It is thus clear that smart choices must be made in what information is communicated between cars. Another, unrelated, issue is the scalability of the network. Scheuermann et al. \cite{Scheuermann2009} show that aggregation of data is a necessity of vehicular ad-hoc networks (VANETs), the predominant communication method for inter-vehicular communication. Furthermore, for any scalable aggregation method, the amount of information that can be received depends on the geographical distance from which it originates in a non-linear manner. In fact, the amount of bandwidth used must be in $o(1/d^2)$, where $d$ is the geographical distance. If this is not the case, then the VANET runs the risk of being overloaded. In practice this means that the obtainable level of detail for information about the road network decreases with distance. It is therefore important to choose carefully what is communicated to agents far away. For instance, it might be important to know about an accident on a road far away, but may not be important to know that the car reporting that is heading west at 60 km/hour.

In addressing what should be communicated we assume drivers are autonomous: there may be decentralized mechanisms in place, such as the use of reputation, to incentivize good behaviour and it may be realistic to assume that extreme misuse of the system could be punished by a central authority in the same way current traffic infractions are punished by the police. However, we do not assume such a system is foolproof and individuals may misuse the communication system for their own benefit. It is up to receiving vehicles to make sense of received information and put it to good use. More on this in the next section; we mentioned it here to emphasize that it is necessary to take into account that what ``should'' be communicated by vehicles may not always be what the vehicles do communicate. Nevertheless, we will assume that the cost of communication is zero. With this we mean that if a vehicle has information that does not affect its own arrival time directly, then the agent does not lose anything by communicating this information to other cars. For instance, if there is a traffic jam on the road going from $A$ to $B$, it costs nothing for cars that are travelling in the opposite direction, from $B$ to $A$, to communicate this. Similarly if there is a patch of ice on the road, cars do not incur a cost broadcasting this information to their neighbours. 

If, on the other hand, the information affects the sender's travel time,  we assume that there is a cost in communicating that is proportional to the expected increase in travel time. For direct communication between vehicles to be beneficial in traffic management we thus need a way to manage it that:
\begin{itemize}
  \item incentivizes local, rather than long-distance communication
  \item incentivizes truthful communication of important information
  \item reduces communication of irrelevant information
  \item incentivizes vehicles to communicate truthfully
\end{itemize}

Computational reputation is a mechanism that could be used for this, but in a massive, decentralized system such as a VANET, where discovering the reputation of a communication partner is itself not without problems \cite{Zhang2011} and current solutions cannot be applied immediately in VANETs. We aim to develop a reputation mechanism that serves two purposes: firstly it allows receivers to decide whether the received information is trustworthy or not, and secondly it incentivizes senders to act in the best interest of the community. 

%In Figure \ref{fig:braess} we draw the road network of the Braess paradox. In the paradox, travel along the roads from $Start$ to $A$ and from $B$ to $Finish$ do not take a fixed amount of time, but are dependent on $T$. $T$ is the total number of cars driving along that road, so if $T = 1000$ then it takes $t = 10$ to drive along the road. The dashed road between $A$ and $B$ may be open or closed and takes a negligible amount of time to drive along. There are a total of 4000 cars driving from $Start$ to $Finish$. 
%\begin{figure}
%\centering
%\begin{tikzpicture}[->,>=stealth',shorten >=1pt,auto,node distance=5cm,main node/.style={circle,draw}]
%
%  \node[main node] (1) {A};
%  \node[main node] (2) [below left=1cm and 4cm of 1] {Start};
%  \node[main node] (3) [below right=1cm and 4cm of 2] {B};
%  \node[main node] (4) [below right=1cm and 4cm of 1] {Finish};
%
%  \path[every node]
%    (1) edge node {$t=45$} (4)
%    	edge [<->,dashed] (3)
%    (2) edge node {$t=\frac{T}{100}$} (1)
%        edge node [swap] {$t=45$} (3)
%    (3) edge node [swap] {$t=\frac{T}{100}$} (4);
%\end{tikzpicture}
%\caption{Road network for the Braess paradox}
%\label{fig:braess}
%\end{figure}
%
%Let us assume the road between $A$ and $B$ is closed. Then both possible routes from start to finish take $t = 45 + \frac{T}{100}$. The optimal action is to choose a path at random. The expected outcome is that half of the cars take the route through $A$ and half take the route through $B$, resulting in $T = 2000$ for both the roads and a total expected travel time of $65$ for all cars. 
%
%The paradox in the network is seen by opening the road between $A$ and $B$. Now the optimal choice is to drive to $B$, take the newly opened road to $A$ and drive to the finish. For both roads $T = 4000$, resulting in a total expected travel time of $80$ for all cars.

\subsection{What should be done with received information?}
The other side of the problem is to consider the information from the receiver's point of view. Upon receiving new information, a driver must decide firstly whether that information is trustworthy or not and secondly whether it is relevant to her current plan. Finally, she may have to replan, taking the new information into account.

There is already a large amount of work on planning in dynamic environments \cite{Buriol2008} and we aim to use that. Rather we focus on the dual problems of deciding what information to trust and how that information should be incorporated into the agent's reasoning model.

These problems have received less attention. They are the reverse side of the same coin we discussed above. When asking "what information should be sent?", the aim is to incentivize the sender to send truthful and useful information. However, the very nature of traffic provides incentives to do exactly the reverse. It is thus up to the receiving agent to distinguish truthful and useful information, from the other received messages.

There are two general approaches to deciding whether information is truthful or not. The first is to evaluate the trustworthiness of the source. The second is to assess the information directly and try to collect corroborating, or contradicting evidence. For instance, if multiple independent sources say the same thing and nobody contradicts it, it is more likely to be true. Neither method is without problem in the context of a VANET.

A large majority of trust models rely on repeat interactions with an individual to learn an evaluation of that individual. In addition, many take communicated evaluations from peers into account, or rely on a centralized source of reputation information. In a road network, we cannot expect any of this information to be available. The chance of a repeat interaction with the same car is small and even finding a driver who has interacted with a specific agent before is small, and if possible will almost invariably take too long for real-time decision making in traffic. A centralized source of reputational information, such as available on websites such as eBay, cannot be accessed efficiently through a VANET and would require a secondary connection through a fixed network to function reliably. Conventional trust models are thus incapable of dealing with the sparse, dynamic environment of traffic. We propose to focus on trust models that can generalize and use stereotypes rather than, or in addition to, individuals \cite{Burnett2010}. For instance, it is likely that traffic heading in the opposite direction will be more trustworthy than traffic heading in the same direction. Machine learning can be used to discern between attributes of senders that are correlated with sending truthful information.

Additionally, the trustworthiness of the message itself can be tested by receiving messages from multiple agents about the same subject. By combining such a comparative approach with a stereotype based trust model we aim to create a robust trust model for application in VANETs.

This trust model evaluates the trustworthiness of a piece of information. However, its relevance to the agent can only be evaluated if we can reason about such concepts as goals. We thus provide to integrate the trust model into a cognitive agent model. The agent's internal reasoning allows for informed decisions about when and where to replan. For instance, if the agent has time to replan before the next major intersection and has received information that its current path is blocked, it is a good time to replan. If the next offramp off a highway is still 10 minutes away, then it is better to wait for new information. This also means the path planning algorithm must be integrated into the cognitive agent model. We propose to use the Possibilistic BDI model \cite{CostaPerreira2010}. This model not only provides a cognitive BDI model, but provides the explicit means for reasoning about the uncertainty inherent in more --- or less --- trusted information.



\section{State of the Art}
\section{Objectives and Workplan}

\bibliographystyle{plain}
\bibliography{proposal}
\end{document}
